\documentclass{beamer}
\mode<presentation>
{
  \usetheme{Warsaw}
  \definecolor{mcgarnet}{rgb}{0.38, 0, 0.08}
  \definecolor{mcgray}{rgb}{0.6, 0.6, 0.6}
  \setbeamercolor{structure}{fg=mcgarnet,bg=mcgray}
  %\setbeamercovered{transparent}
}


\usepackage[english]{babel}
\usepackage[latin1]{inputenc}
\usepackage{times}
\usepackage[T1]{fontenc}
\usepackage{tikz}
\usepackage{graphicx}

\newcommand{\imagesource}[1]{{\centering\hfill\break\hbox{\scriptsize \thinspace{\small\itshape #1}}\par}}

\title{Advanced Magic - GDB}


\author{Robert Lowe\\}

\institute[Maryville College] % (optional, but mostly needed)
{
  Division of Mathematics and Computer Science\\
  Maryville College
}

\date[]{}
\subject{}

\pgfdeclareimage[height=0.5cm]{university-logo}{images/Maryville-College}
\logo{\pgfuseimage{university-logo}}



\AtBeginSection[]
{
  \begin{frame}<beamer>{Outline}
    \tableofcontents[currentsection]
  \end{frame}
}


\begin{document}

\begin{frame}
  \titlepage
\end{frame}

\begin{frame}{Outline}
  \tableofcontents
\end{frame}


% Structuring a talk is a difficult task and the following structure
% may not be suitable. Here are some rules that apply for this
% solution: 

% - Exactly two or three sections (other than the summary).
% - At *most* three subsections per section.
% - Talk about 30s to 2min per frame. So there should be between about
%   15 and 30 frames, all told.

% - A conference audience is likely to know very little of what you
%   are going to talk about. So *simplify*!
% - In a 20min talk, getting the main ideas across is hard
%   enough. Leave out details, even if it means being less precise than
%   you think necessary.
% - If you omit details that are vital to the proof/implementation,
%   just say so once. Everybody will be happy with that.
\section{Introduction to GDB}
\begin{frame}
    \frametitle{What is a debugger?}
    \begin{columns}
        \column{0.7\textwidth}
        \begin{itemize}[<+->]
            \item A debugger is a program that helps you
                find and correcting errors in programs.
            \item Debuggers typically:
            \begin{itemize}
                \item Allow programs to be stopped at
                    fixed locations.
                \item Step through a program.
                \item Display program state.
                \item Allow program state to be altered.
                \item Allow post-mortem examination of a
                    program's memory.
            \end{itemize}
        \end{itemize}
        \column{0.3\textwidth}
        \includegraphics[width=\textwidth]{images/archer}
        \imagesource{www.gnu.org}
    \end{columns}
\end{frame}

\begin{frame}
    \frametitle{Getting Started}
    \begin{itemize}[<+->]
       \item Before a program can be effectively debugged, we must put symbols\footnotemark in the object file by using the {\tt -g} option. 
           \par{\tt g++ -g program.cpp -o program}
       \item In a makefile, you can specify this with the
           {\tt CXXFLAGS} variable.
           \par{\tt CXXFLAGS=-g}
       \item If you are using implicit makefile rules directly (without a makefile), you can specify {\tt -g} by setting an environment variable.
           \par{\tt export CXXFLAGS=-g}
       \item You invoke the debugger like this:
           \par{\tt gdb program}
    \end{itemize}
    \footnotetext[1]{You can debug without symbols, but it will be in machine code!}
\end{frame}

\begin{frame}
    \frametitle{Listing Programs}
    \begin{description}[<+->]
        \item[list] List 10 lines of the program.  Repeating this command lists the next 10 lines.
        \item[list {\it num}] List 10 lines with line number {\it num} in the middle.
        \item[list {\it function}] List the source code of a given function.
        \item[list {\it classname}::{\it function}] List 
        the source code of a given class member function.
        \item[list {\it source.cpp}:1] List the contents of
        the source file {\it source.cpp} beginning at line 1.
    \end{description}
\end{frame}

\begin{frame}
    \frametitle{Basics of the GDB User Interface}
    \begin{itemize}[<+->]
        \item Pressing enter without typing a command repeats the last command. 
        \item Most commands have abbreviations.  Generally, this is the first letter of the command. (For example, {\tt list} is abbreviated as {\tt l})
        \item gdb has an extensive help system.  This is accessed by the "help" command.  Try "help list".
        \item Typing "help" by itself lists broad help categories.
        \item You can scroll throw previous commands using your arrow keys.
    \end{itemize}
\end{frame}

\section{Working with Breakpoints and Stepping Programs}
\begin{frame}
    \frametitle{Breakpoint Commands}
    A {\bf breakpoint} is a location in a program where the
    debugger will pause execution so you can have a look around.
    \begin{description}[<+->]
        \item[break {\it num}] Set a breakpoint at line {\it num}.
        \item[break {\it num} if {\it condition}] Set a conditional breakpoint at line {\it num}.  The breakpoint stops only if the {\it condition} is met.  The condition follows the same syntax as C++ boolean expressions.
        \item[info breakpoints] Display the list of 
        breakpoints.
        \item[delete breakpoints {\it num}] Delete breakpoint {\it num}.
        \item[delete breakpoints] Delete all breakpoints.
        \item[break *{\it address}] Advanced wizards only!  Break when execution reaches the specified memory address.
    \end{description}
\end{frame}

\begin{frame}
    \frametitle{Running, Continuing, and Stopping Programs}
    Once you have your breakpoints set up, it is time to run the program!
    \begin{description}[<+->]
        \item[run] Run the current program.  If the program is already running, it asks if you want to restart the program from the beginning.
        \item[kill] Terminate the currently running program.
        \item[continue] Continue execution of a paused program.
        \item[Control + C] Pause the execution of a currently running program.  (Like a breakpoint, but from a key combination!)
    \end{description}
\end{frame}

\begin{frame}
    \frametitle{Next and Step}
    When a program is stopped, either at a breakpoint or 
    through use of Control + c, you can step through it 
    one line at a time.
    \begin{description}[<+->]
        \item[next] Execute the current line, stopping on the next line of code in the current function.  This command does not go into a function.
        \item[step] Execute the current line, stopping on the next line of code.  If the next line of code is in a different function, step will go into the function and stop at its first line.  The name implies that it "steps into" functions.
        \item[finish] Continue execution until just after the current function returns.
    \end{description}
\end{frame}

\section{Displaying Data}
\begin{frame}
    \frametitle{Print and Display}
    \begin{description}[<+->]
        \item[print {\it exp}] Print the value of the expression one time.  
        \item[display {\it exp}] Print the value of the expression every time the program stops.
        \item[delete display {\it num}] Delete the display numbered {\it num}
        \item[delete display] Delete all displays
        \item[info display] List all auto-display expressions.
        \item[disable display {\it num}] Disable display number {\it num}.
        \item[disable display] Disable all displays.
        \item[enable display {\it num}] Enable display {\it num}.
        \item[enable display] Enable all display instructions.
    \end{description}
\end{frame}


\begin{frame}
    \frametitle{Evaluating Expressions}
    GDB expressions are processed in much the same way as 
    c++ expressions.  They can contain:
    \begin{itemize}[<+->]
        \item Variable Names
        \item Literal Values
        \item Arithmetic Operations
        \item Index Operations
        \item Dereference Operations
        \item Address of Operations
        \item Function Calls
    \end{itemize}
    
    \begin{description}[<+->]
        \item[call {\it exp}] Call an expression.  This can handle void functions.  If the function returns something, the result is displayed.
    \end{description}
\end{frame}

\begin{frame}
    \frametitle{Examining the Call Stack}
    gdb provides a lot of handlers for exploring the 
    function call stack.
    \begin{description}[<+->]
        \item[backtrace] List all stack frames up to this point in the code.
        \item[frame {\it num}] Switch the debugger's context to the stack frame indicated by {\it num}. 
        \item[up] Switch to the previous stack frame.
        \item[down] Switch to the next stack frame.
    \end{description}
\end{frame}

\section{Low Level Concepts}

\begin{frame}
    \frametitle{Examining Registers}
    Registers are temporary holding locations in the CPU
    which are used as a sort of "scratch pad" for programs to work with.  Some have special purposes.  GDB lets us examine them!
    \begin{description}[<+->]
        \item[info registers] Display all the registers.
        \item[print \${\it register}] Print the register.
        \item[display \${\it register}] Display the register.
        \item[\${\it register}] Is generally how we use registers in expressions in gdb.
    \end{description}
\end{frame}

\begin{frame}
    \frametitle{Machine Code Commands}
    \begin{description}[<+->]
        \item[nexti] Execute the current machine code
        instruction, stopping at the next one.  Does not step into subroutines.
        \item[stepi] Execute the current machine code instruction, stopping at the next one.  Steps into subroutines.
        \item[x/{\it fmt} {\it address}] Examine the raw memory beginning at address, applying the given format.  
            \begin{enumerate}
                \item Repeat Count
                \item Format letter: o(octal), x(hex), d(decimal), u(unsigned decimal), t(binary), f(float), a(address), i(instruction), c(char), s(string)
                \item Size Letter: b(byte), h(halfword), w(word), g(giant, 8 bytes)
            \end{enumerate}
    \end{description}
\end{frame}


\end{document}


