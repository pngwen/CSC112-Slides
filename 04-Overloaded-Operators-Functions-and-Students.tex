\documentclass{beamer}
\mode<presentation>
{
  \usetheme{Warsaw}
  \definecolor{mcgarnet}{rgb}{0.38, 0, 0.08}
  \definecolor{mcgray}{rgb}{0.6, 0.6, 0.6}
  \setbeamercolor{structure}{fg=mcgarnet,bg=mcgray}
  %\setbeamercovered{transparent}
}


\usepackage[english]{babel}
\usepackage[latin1]{inputenc}
\usepackage{times}
\usepackage[T1]{fontenc}
\usepackage{tikz}
\usepackage{graphicx}

\newcommand{\imagesource}[1]{{\centering\hfill\break\hbox{\scriptsize Image Source:\thinspace{\small\itshape wikipedia.org}}\par}}

\title{Overloaded Operators, Functions, and Students}


\author{Robert Lowe\\}

\institute[Maryville College] % (optional, but mostly needed)
{
  Division of Mathematics and Computer Science\\
  Maryville College
}

\date[]{}
\subject{}

\pgfdeclareimage[height=0.5cm]{university-logo}{images/Maryville-College}
\logo{\pgfuseimage{university-logo}}



\AtBeginSection[]
{
  \begin{frame}<beamer>{Outline}
    \tableofcontents[currentsection]
  \end{frame}
}


\begin{document}

\begin{frame}
  \titlepage
\end{frame}

\begin{frame}{Outline}
  \tableofcontents
\end{frame}


% Structuring a talk is a difficult task and the following structure
% may not be suitable. Here are some rules that apply for this
% solution: 

% - Exactly two or three sections (other than the summary).
% - At *most* three subsections per section.
% - Talk about 30s to 2min per frame. So there should be between about
%   15 and 30 frames, all told.

% - A conference audience is likely to know very little of what you
%   are going to talk about. So *simplify*!
% - In a 20min talk, getting the main ideas across is hard
%   enough. Leave out details, even if it means being less precise than
%   you think necessary.
% - If you omit details that are vital to the proof/implementation,
%   just say so once. Everybody will be happy with that.
\section{Overloading Symbols}
\begin{frame}
    \frametitle{Symbol Overloading}
    
    \begin{itemize}
        \item<2-> A symbol is {\bf overloaded} if it has multiple meanings.
        \item<3-> Overloading is frequently called "ad-hoc polymorphism".
        \item<4-> Also referred to as syntactic sugar.
        \item<5-> Overloading is meant to keep code close to the actual
          program domain.
        \item<6-> Establishes different behaviors while allowing the 
          programmer to express intent.
    \end{itemize}
\end{frame}


\begin{frame}
    \frametitle{Overloaded Function Example: {\tt abs}}
    \uncover<2->{
        {\bf In C} 
        \par {\tt int abs(int j);}
        \par {\tt double fabs(double x);}
    }
    \vspace{0.25in} 
    \par\uncover<3->{
        {\bf In C++}
        \par {\tt int abs(int j); }
        \par {\tt double abs(double x);}
    }
\end{frame}

\begin{frame}
    \frametitle{C++ Function Signature}
    {\bf A Normal Function Signature Consists Of}
    \begin{itemize}
        \item<2-> Name
        \item<3-> Number of Parameters
        \item<4-> Parameter Types
    \end{itemize}
    
    \begin{block}{C++ Overloading Rules}
        In C++, overloaded functions share the same name.  They
        must vary in at least one other part of the normal signature.
    \end{block}
\end{frame}

\begin{frame}[fragile]
    \frametitle{The Overloaded Accessor/Mutator Idiom}
    \begin{verbatim}
std::string firstName();
void firstName(const std::string & _firstName);
    \end{verbatim}
    
    \begin{itemize}[<+->]
        \item Both functions have the same name {\tt firstName}
        \item Both functions differ by number of arguments.
        \item The compiler uses the function signature to determine which
          function to execute.
        \item Reusing the same name means less information for the
          the programmer to remember!
    \end{itemize}
\end{frame}

\section{Overloaded Operators}
\begin{frame}
    \frametitle{Operators as Functions}
    \begin{itemize}[<+->]
        \item An operator could be viewed as a function!
        \item A binary operator takes two arguments:
            \par{\tt a + b == add(a, b)}
        \item A unary operator takes one argument:
            \par{\tt ++a == increment(a)}
        \item Like functions, operators can be overloaded.
        \item Existing operators cannot be overridden, but you can 
            overload types so long as they use at least one class or 
            enumerated type.
    \end{itemize}
\end{frame}

\begin{frame}[fragile]
    \frametitle{Operator Overloading Syntax}
    {\tt {\em <return\_type>} operator{\em <symbol>}({\em <parameters>})}
    \vspace{0.25in}
    \par {\bf Example an Overloaded < Operator}
    \begin{verbatim}
bool operator<(const PhoneRecord &lhs, 
               const PhoneRecord &rhs)
{
    if(lhs.lastName() < rhs.lastName()) {
        return true;
    }
    
    return lhs.firstName() < rhs.firstName();
}
    \end{verbatim}
\end{frame}

\begin{frame}[fragile]
    \frametitle{Overloading Arithmetic Operators}
    \begin{itemize}
        \item \begin{verbatim}
type operator+(const type &lhs, 
               const type &rhs);\end{verbatim}
        \item \begin{verbatim}
type operator-(const type &lhs, 
               const type &rhs);\end{verbatim}
        \item \begin{verbatim}
type operator*(const type &lhs, 
               const type &rhs);\end{verbatim}
        \item \begin{verbatim}
type operator/(const type &lhs, 
               const type &rhs);\end{verbatim}
        \item \begin{verbatim}
type operator%(const type &lhs, 
               const type &rhs);\end{verbatim}
    \end{itemize}
\end{frame}

\begin{frame}[fragile]
    \frametitle{Overloading Increment and Decrement Operators}
    {\bf Prefix Operators {\tt ++x;}}
    \begin{itemize}
        \item \begin{verbatim}type & operator++(type & lhs);\end{verbatim}
        \item \begin{verbatim}type & operator--(type & lhs);\end{verbatim}
    \end{itemize}
    \vspace{0.25in}
    \par{\bf Postfix Operators {\tt x++;}}
    \begin{itemize}
        \item \begin{verbatim}type operator++(type & lhs, int);\end{verbatim}
        \item \begin{verbatim}type operator--(type & lhs, int);\end{verbatim}
    \end{itemize}
\end{frame}

\begin{frame}[fragile]
    \frametitle{Overloading Equality and Comparison Operators}
    \begin{itemize}
        \item \begin{verbatim}
bool operator==(const type &lhs, 
                const type &rhs);\end{verbatim}
        \item \begin{verbatim}
bool operator!=(const type &lhs, 
                const type &rhs);\end{verbatim}
        \item \begin{verbatim}
bool operator<(const type &lhs, 
               const type &rhs);\end{verbatim}
        \item \begin{verbatim}
bool operator<=(const type &lhs, 
                const type &rhs);\end{verbatim}
        \item \begin{verbatim}
bool operator>(const type &lhs, 
               const type &rhs);\end{verbatim}
        \item \begin{verbatim}
bool operator>=(const type &lhs, 
                const type &rhs);\end{verbatim}
    \end{itemize}
\end{frame}

\begin{frame}[fragile]
    \frametitle{Overloading Assignment Operators}
    \begin{itemize}
        \item \begin{verbatim}
type operator=(type &lhs, const type &rhs);\end{verbatim}
        \item \begin{verbatim}
type operator+=(type &lhs, const type &rhs);\end{verbatim}
        \item \begin{verbatim}
type operator-=(type &lhs, const type &rhs);\end{verbatim}
        \item \begin{verbatim}
type operator*=(type &lhs, const type &rhs);\end{verbatim}
        \item and so on...
    \end{itemize}
\end{frame}

\begin{frame}[fragile]
    \frametitle{Overloading Index Operators}
    \begin{itemize}[<+->]
        \item Often we want to create indexable containers. 
        \item Consider vector indexing: {\tt v[0];}
        \item This is an overloaded operator!:
        \begin{verbatim}
type & operator[](type &lhs, int index);
        \end{verbatim}
    \end{itemize}
\end{frame}

\begin{frame}[fragile]
    \frametitle{Overloaded Stream Operators}
    {\bf Insertion Operator}
    \begin{verbatim}
ostream &operator<<(ostream &os, const type &rhs);
    \end{verbatim}
    \vspace{0.10in}
    \par{\bf Extraction Operator}
    \begin{verbatim}
istream &istream>>(istream &is, const type &rhs);
    \end{verbatim}
\end{frame}

\begin{frame}[fragile]
    \frametitle{Overloaded Operators as Member Methods}
    \begin{itemize}[<+->]
        \item An overloaded operator can be a member method!
        \item When an overloaded operator is a member method, the left
          hand side is the current object. 
        \item Only the right hand side is specified in the parameter list.
        \item Example: The {\tt PhoneRecord}'s index operator.
            \begin{verbatim}
PhoneNumber & operator[](int index);
            \end{verbatim}
    \end{itemize}
\end{frame}

\section{Best Practices}
\begin{frame}
    \frametitle{Don't Mislead Users of Your Classes!}
    \begin{itemize}[<+->]
        \item Overloaded operators are meant to make code match its
            problem domain.
        \item Your overloaded operators should match the original 
            purpose of the operator.
        \item Avoid side effects, unless they are expected.
        \item If an operator doesn't have an obvious relationship to 
            your class's operations, don't overload it!
    \end{itemize}
\end{frame}

\begin{frame}
    \frametitle{Meet User's Expectations!}
    \begin{itemize}[<+->]
        \item It is a good idea to overload at least assignment operators.
        \item If your class is likely to have objects in a container, provide
            at least {\tt == } and {\tt <} operators.
        \item Insertion and Extraction Operators are sometimes expected,
            but don't use them in situations where the I/O of the object
            is not well defined.
        \item If you do provide insertion and extraction operators, the
            insertion operator's output should be valid input for the
            extraction operator.
    \end{itemize}
\end{frame}

\begin{frame}
    \frametitle{Keep it Inside the Class.}
    \begin{itemize}[<+->]
        \item Unless its impossible, operators should be member classes.
        \item Operators that do not change the current object should be 
            marked {\tt const}.
        \item Avoid the use of {\tt friend} for insertion and extraction
            operators.
        \item If you find you must make a stream operator a {\tt friend},
            chances are your class API is incomplete!
    \end{itemize}
\end{frame}
\end{document}


