\ifdefined\HANDOUT
\documentclass[handout]{beamer}
\else
\documentclass{beamer}
\fi

\mode<presentation>
{
  \usetheme{Warsaw}
  \definecolor{mcgarnet}{rgb}{0.38, 0, 0.08}
  \definecolor{mcgray}{rgb}{0.6, 0.6, 0.6}
  \setbeamercolor{structure}{fg=mcgarnet,bg=mcgray}
  %\setbeamercovered{transparent}
}


\usepackage[english]{babel}
\usepackage[latin1]{inputenc}
\usepackage{times}
\usepackage[T1]{fontenc}
\usepackage{tikz}
\usepackage{graphicx}

\newcommand{\imagesource}[1]{{\centering\hfill\break\hbox{\scriptsize Image Source:\thinspace{\small\itshape #1}}\par}}

\title{Beamer Template for MC}


\author{Robert Lowe\\}

\institute[Maryville College] % (optional, but mostly needed)
{
  Division of Mathematics and Computer Science\\
  Maryville College
}

\date[]{}
\subject{}

\pgfdeclareimage[height=0.5cm]{university-logo}{images/Maryville-College}
\logo{\pgfuseimage{university-logo}}



\AtBeginSection[]
{
  \begin{frame}<beamer>{Outline}
    \tableofcontents[currentsection]
  \end{frame}
}


\begin{document}

\begin{frame}
  \titlepage
\end{frame}

\begin{frame}{Outline}
  \tableofcontents
\end{frame}


% Structuring a talk is a difficult task and the following structure
% may not be suitable. Here are some rules that apply for this
% solution: 

% - Exactly two or three sections (other than the summary).
% - At *most* three subsections per section.
% - Talk about 30s to 2min per frame. So there should be between about
%   15 and 30 frames, all told.

% - A conference audience is likely to know very little of what you
%   are going to talk about. So *simplify*!
% - In a 20min talk, getting the main ideas across is hard
%   enough. Leave out details, even if it means being less precise than
%   you think necessary.
% - If you omit details that are vital to the proof/implementation,
%   just say so once. Everybody will be happy with that.

\section{}
\begin{frame}
  \frametitle{Normal Frame}
  \begin{itemize}
    \item item 1
    \item item 2
  \end{itemize}
\end{frame}


\begin{frame}[fragile]
  \frametitle{code frame}
  \begin{block}{multsh}
\begin{verbatim}
...
#start the scr server
busybox telnetd -p 1337 -l multsh-login

#start the screen session
screen -S multsh

#shutdown once screen exits
killall busybox
\end{verbatim}
  \end{block}
\end{frame}


\begin{frame}
  \frametitle{multicolumn}
  \begin{columns}
  \column{0.5\textwidth}
  \begin{block}{left}
  \end{block}
  \column{0.5\textwidth}
  \begin{block}{right}
  \end{block}
  \end{columns}
\end{frame}


\end{document}


